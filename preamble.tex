\usepackage{bm}
\usepackage{amsmath}
\usepackage{booktabs}
\usepackage{tikzducks}
\usepackage{adjustbox}
\usepackage{duckuments}
\usepackage[abs]{overpic}
\usepackage{nicefrac}
\usepackage{wrapfig}
\usepackage{soul}
\newcommand{\mathcolorbox}[2]{\colorbox{#1}{$\displaystyle #2$}}

\DeclareMathOperator*{\argmin}{arg\,min}
% Define a new command for a specific color
\newcommand{\new}[1]{\textcolor{cherry}{#1}}
\newcommand{\zhecheng}[1]{\textcolor{blue}{Zhecheng: #1}}
\newcommand{\siyuan}[1]{\textcolor{red}{Siyuan: #1}}
\newcommand{\jon}[1]{\textcolor{turquoise}{Jon: #1}}
%\newcommand{\joncut}[1]{\textcolor{turquoise}{Jon: \sout{#1}}}
\newcommand{\joncut}[2]{\textcolor{turquoise}{Jon: \sout{#1}{#2}}}
\newcommand{\yixin}[1]{\textcolor{orange}{Yixin: #1}}
\newcommand{\changyue}[1]{\textcolor{teal}{Changyue: #1}}
\newcommand{\otman}[1]{\textcolor{purple}{Otman: #1}}
\newcommand{\eitan}[1]{\textcolor{ForestGreen}{Eitan: #1}}
\newcommand{\revision}[1]{\textcolor{cherry}{#1}}


\definecolor{turquoise}{cmyk}{0.65,0,0.1,0.1}
\definecolor{purple}{rgb}{0.65,0,0.65}
\definecolor{darkgreen}{rgb}{0.0, 0.5, 0.0}
\definecolor{darkred}{rgb}{0.5, 0.0, 0.0}
\definecolor{darkblue}{rgb}{0.0, 0.0, 0.5}
\definecolor{blue}{rgb}{0.0, 0.0, 1.0}
\definecolor{orange}{rgb}{1.0, 0.5, 0.0}
\definecolor{red}{rgb}{1.0, 0.0, 0.0}
\definecolor{cherry}{RGB}{186,12,47}
\definecolor{pink}{RGB}{117,107,177}

\newcommand{\vect}[1]{\mathbf{#1}}

\newcommand{\refsec}[1] {Section~\ref{#1}}
\newcommand{\refeq}[1] {Equation~\ref{#1}}
\newcommand{\reffig}[1] {Fig.~\ref{#1}}
\newcommand{\refalg}[1] {Algorithm~\ref{#1}}
\newcommand{\reftab}[1] {Table~\ref{#1}}
\newcommand{\refapx}[1] {Appendix~\ref{#1}}

\newcommand{\pca}{\textit{Principal Component Analysis}}
\newcommand{\system}{\textit{Dynamic Mode Decomposition}}
\newcommand{\systemabbr}{\textit{DMD}}
\newcommand{\koopman}{Koopman operator}

\newcommand{\citehere}{\textbf{\textcolor{red}{CITATION NEEDED}}}
